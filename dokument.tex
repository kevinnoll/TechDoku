\documentclass[a4paper,abstracton]{scrreprt}
\usepackage[T1]{fontenc}
\usepackage[utf8]{inputenc}
\usepackage[ngerman]{babel}

%correct linebreaking in bibliography
\usepackage{hyperref}
\usepackage{breakurl}

%lists
\usepackage{mdwlist}

%biblatex
%\usepackage[babel,german=quotes]{csquotes}
%\usepackage[style=alphabetic]{biblatex}
%\bibliography{thesis}



%table 
\usepackage{pbox}
\usepackage{booktabs}

%set numeration depth
\setcounter{secnumdepth}{3}
%set how many numbers show up in table-of-contents
\setcounter{tocdepth}{2}

\begin{document}
\author{Kevin Noll}
\subject{Pilze und so}
\title{Roetlinge, yo}
\publishers{htwsaar}
\maketitle
\tableofcontents
%\listoffigures
%\listoftables

\begin{abstract}
\begin{quote}%abstand rechts und links
Diese Arbeit befasst sich mit Pizlen und so nem kram. kein scheiss
\end{quote} 
\end{abstract}

\chapter{Bodenbeschaffenheit}
\chapter{Beschreibung der Pilze}
\chapter{Verwechslungsmoeglichkeiten}
\chapter{Inhaltsstoffe, Geniessbarkeit}
\chapter{Ernte, Haltbarkeit und richtige Lagerung}
\chapter{Verwendung und Zubereitung mit Rezept}

%\printbibliography
\end{document}